Carbon uptake by terrestiral plants is a major determinant of Earth's climate system (say more precisely). In mid and high latitudes, the net carbon uptake is primiarily determined by the length of the growing season. Climate change has advanced spring leafout, widely considred the start of the growing season. It has been widely suggest that earlier start of season should translate into a longer growing season and, therefore, more net carbon uptake by terrestrial ecosystems, but recent evidence has cast a doubt of this assumption. Quantifying carbon uptake by terrestrial plants is a critical aspect of climate forecasting, and therefore it is critical to understand the relationship between spring phenology and growing season length.

This relationship hinges on the what is happening at the end of season, and our understanding of control of fall phenology are far less developed. Spring phenology is largely determined by three environemntal cues with temperature being the dominant. There are 3 hypotheses about the controlls of fall phenology that translate directly into determining growing season length.

1. Fall phenology is also under strong temperature control. Climate warming should really extend the growing season.

2. Fall phenology is under photoperiod control, therefore less plastic. Marginal extention of growing season due to spring advance.

3. Sink hypothesethis from Renner. Fall phenology is determined by aquisition of carbon resources. Therefore spring and fall phenology are positively correlated, and growing season is relatively stable.


These hypotheses have been investigated at the ecosystem scale (remote sensing) which may miss the critical local scale mechanisms driving these patterns. Since ecosystem functions like carbon storage are the product of the unique assembly of species present in a given community, it is valuable to investigate the relationship between spring phenology at the species and population level. Common garden studies offer potential to assess the relationship between SoS and EoS phenological events, and how they vary in ecological communities. However, multi-species common garden are relatively rare, so it is difficult to build up a community context (say better).

In this study, we leverage three years worth of full season phenological observations from a 12 species common garden study of 4 populations Northeastern North America to assess:

1. Variation in growith season length amoung species and population under shared environmental conditions.

2. The relationship between SoS variability and growing season length.

This study offers insights into physiology that will allow us to scale from ecosystem level observations to individual mechansims woo, and improve forecasting.